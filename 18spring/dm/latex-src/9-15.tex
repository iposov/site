\documentclass{article}

\usepackage[utf8]{inputenc}
\usepackage[russian]{babel}
\usepackage{amsmath}

\newcommand{\m}[9]{
$$
\begin{array}{|c|c|c|}
\hline
#1&#2&#3\\
\hline
#4&#5&#6\\
\hline
#7&#8&#9\\
\hline
\end{array}
$$
}

\begin{document}
	\m{}{}{} {}{15}{} {9}{}{}
	
	\m{}{}{x} {}{15}{} {9}{}{}
	
Сумма линий в магическом квадрате теперь равна $9 + 15 + x=24 + x$.
	
	\m{}{}{x} {y}{15}{} {9}{}{}
	
Первый столбец даёт:
	\m{15+x-y}{}{x} {y}{15}{} {9}{}{}

Вторая строка даёт:
    \m{15+x-y}{}{x} {y}{15}{9+x-y} {9}{}{}

Главная диагональ даёт:
    \m{15+x-y}{}{x} {y}{15}{9+x-y} {9}{}{-6+y}
    
Сумма в третьем столбце должна быть равна $24+x$:
$$24+x = x + (9 + x - y) + (-6 + y) = 3 + 2x.$$
Отсюда
$$x=21.$$ Сумма чисел в линиях квадрата, соответственно $24 + 21=45$. Подставим это в последнюю таблицу:

\m{36-y}{}{21} {y}{15}{30-y} {9}{}{-6+y}

Первая строка:
\m{36-y}{-12+y}{21} {y}{15}{30-y} {9}{}{-6+y}

Третья строка:
\m{36-y}{-12+y}{21} {y}{15}{30-y} {9}{42-y}{-6+y}

Теперь все сходится, квадрат магический, это не зависит от $y$. Если выбрать $y=13$, получится квадрат:

\m{23}{1}{21} {13}{15}{17} {9}{29}{7}

\end{document}