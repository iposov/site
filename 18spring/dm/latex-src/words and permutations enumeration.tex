\documentclass{article}

\usepackage[utf8]{inputenc}
\usepackage[russian]{babel}
\usepackage{amsmath}
\usepackage{cancel}

\newenvironment{exercise}{%
\begin{framed}\par\noindent\slshape%
}%
{\end{framed}}

\title{Перечисление слов и перестановок}
\author{}
\date{}

\begin{document}
\maketitle

\section{Перечисление слов}
Рассмотрим все слова длины 5 над алфавитом $A = \{a,b,c\}$. Всего таких слов $3^5=243$, потому что имеется пять позиций по три возможных буквы в каждой.

Выпишем все слова по алфавиту:

\begin{center}
\begin{tabular}{cc}
	номер & слово \\
	\hline
	1 & \verb|aaaaa| \\
	2 & \verb|aaaab| \\
	3 & \verb|aaaac| \\
	4 & \verb|aaaba| \\
	\multicolumn{2}{c}{$\cdots$} \\
	242 & \verb|ccccb| \\
	243 & \verb|ccccc|
\end{tabular}
\end{center}

Задача состоит в том, чтобы, зная слово, определить его номер или, зная номер, определить слово. Например, если требуется найти слово с номером 1, ответом будет \verb|aaaaa|, а если требуется найти номер слова \verb|ccccc|, то ответом будет 243.

Для решения задачи нужно заметить буквы в алфавитном порядке на целые числа от 0 и выше: $a\rightarrow0$, $b\rightarrow1$, ...
Тогда оказывается, что слова в алфавитном порядке представляют собой подряд идущие числа в троичной системе счисления:

\begin{center}
	\begin{tabular}{ccccc}
		номер & слово & & троичное число & десятичное число \\
		\hline
		1 & \verb|aaaaa| & $\rightarrow$ & \verb|00000| & 0 \\
		2 & \verb|aaaab| & $\rightarrow$ & \verb|00001| & 1 \\
		3 & \verb|aaaac| & $\rightarrow$ & \verb|00002| & 2 \\
		4 & \verb|aaaba| & $\rightarrow$ & \verb|00010| & 3 \\
%		\multicolumn{5}{c}{$\cdots$} \\
        $\cdots$&$\cdots$&&$\cdots$&$\cdots$\\
		242 & \verb|ccccb| & $\rightarrow$ & \verb|22221| & 241 \\
		243 & \verb|ccccc| & $\rightarrow$ & \verb|22222| & 242
	\end{tabular}
\end{center}

Естественно, в случае другого алфавита система счисления может быть не троичной.

В таблице видно, что число, получающееся из слова заменой букв на цифры, на единицу отличается от номера слова. Про это нельзя забывать, иначе при решении задач ответ может отличаться от правильного на единицу.

Соответственно, чтобы по номеру определить слово, нужно
\begin{enumerate}
	\item вычесть единицу;
	\item перевести в троичную систему счисления;
	\item заменить цифры на буквы.
\end{enumerate}

Чтобы по слову узнать номер, нужно проделать эти действия в обратном порядке.

Например, для номера 42 получается следующие действия
\begin{enumerate}
	\item 41;
	\item \verb|1112|;
	\item \verb|abbbc| (не \verb|bbbc|, т.к. в задаче рассматриваются слова длины 5).
\end{enumerate}

\section{Перечисление перестановок}
Рассмотрим перестановки чисел от 1 до 5 и расположим их в алфавитном порядке.

\begin{center}
	\begin{tabular}{cc}
		номер & перестановка \\
		\hline
		1 & \verb|12345| \\
		2 & \verb|12354| \\
		3 & \verb|12435| \\
		4 & \verb|12453| \\
		\multicolumn{2}{c}{$\cdots$} \\
		119 & \verb|54312| \\
		$5!=120$ & \verb|54321|
	\end{tabular}
\end{center}

Попробуем определить номер перестановки \verb|24531|. Если вы хотите только научиться решать задачу, можете перейти к концу раздела, где приведен алгоритм. Сейчас будут приведены рассуждения, которые объясняют, почему алгоритм работает.

Все первые перестановки начинаются с единицы. Сколько перестановок начинается с единицы? $4!=24$. Дальше все перестановки начинаются с двойки, их тоже 24. Так как искомая перестановка начинается с двойки, ее номер должен получиться от 25 до 48.

Рассмотрим теперь только перестановки, начинающиеся с двойки. Если найти среди них номер перестановки \verb|24531|, для окончательного ответа к нему необходимо будет прибавить 24.

Среди перестановок, начинающихся с \verb|2|, сначала идут перестановки, начинающиеся на \verb|21|, их $3!=6$ штук, потом перестановки, начинающиеся на \verb|23|, их тоже 6 штук, их номера с 7 по 12. Дальше идут 6 перестановок, начинающихся на \verb|24|, их тоже 6 штук, их номера с 13 по 18. Искомая перестановка как раз начинается с \verb|24|, поэтому пока остановимся.

Рассмотрим теперь только перестановки, начинающиеся на \verb|24| и найдем среди них исходную. Если к ее номеру добавить $2\times6$, а потом 24, то получится окончательный ответ. Есть две ($2! = 2$) перестановки, начинающиеся
на \verb|241|, потом две перестановки на \verb|243|, потом две перестановки, начинающиеся на \verb|245|. Соответственно, искомая перестановка находится в последней группе, ее окончательный номер будет равен $24 + 2\times6+2\times2$ плюс номер, среди перестановок, начинающихся на \verb|245|.

Рассуждения можно продолжить дальше, но в этом примере можно остановиться уже сейчас, потому что видно, что искомая перестановка вторая в группе, начинающихся на \verb|245|, поэтому окончательный ответ равен $24 + 2\times6+2\times2 + 2 = 42$, т.е. номер перестановки \verb|24531| равен 42.

%Попробуем определить перестановку по номеру 42. Если вы хотите только научиться решать задачу, можете перейти к концу раздела, где приведен алгоритм. Сейчас будут приведены рассуждения, которые объясняют, почему алгоритм работает.

%Все первые перестановки начинаются с единицы. Сколько перестановок начинается с единицы? $4!=24$. Дальше все перестановки начинаются с двойки, их тоже 24. Получается, что первые 48 перестановок начинаются с единицы или двойки, а перестановка с номером 42 начинается с двойки.

%Среди 24 перестановок, начинающихся с двойки, нас интересует перестановка с номером $42-24=18$. Если известно, что первая цифра равна двум, остальные цифры перестановки~--- это 1, 3, 4, 5.

%Количество перестановок, начинающихся на \verb|21|, равно $3!=6$, соответственно, первые 6 перестановок начинаются на \verb|21|, следующие 6 перестановок (c 7 по 12) начинаются на \verb|23|, следующие 6 (с 13 по 18) перестановок начинаются с \verb|24|, оставшиеся 6 перестановок (с 19 по 24) начинаются на \verb|25|. Получается, что 18 перестановка начинается с четверки.

%Известно, что перестановка начинается с \verb|24|, при этом это 18-ая из тех, которые начинаются с \verb|2|, значит, она $18-6-6=6$, шестая из тех, которые начинаются на \verb|24|. Можно пойти дальше тем же путем и разбираться, какая в перестановке следующая цифра, или в конкретно этом примере можно получить ответ уже сейчас, т.к. 6-ая перестановка это последняя из шести, значит, цифры 1 3 5 идут в ней в убывающем порядке, и ответ получается \verb|24531|.

%Приведенные выше рассуждения объясняют, как получить 

\subsection{Вычисление номера перестановки}

Проделаем те же самые действия для перестановки \verb|24531| теперь в виде алгоритма.

Сначала перестановку нужно превратить в набор чисел, который называется таблицей инверсий. Для каждого числа в перестановке пишется количество чисел, которые находятся правее, и которые меньше этого числа. Для двойки нужно написать 1, потому что есть только единица справа от двойки, меньшая двойки. Дальше для четверки нужно написать 2, потому что справа от четверки есть два меньших числа, это 3 и 1. И т.д. В результате получается таблица инверсий \verb|12210|. Обратите внимание, что в таблице инверсий последнее число всегда 0, потому что правее правого ничего нет.

После этого выписываем ответ, нужно домножить факториалы на числа из таблицы инверсий, начиная с 4! для перестановки из 5 элементов (в общем случае, если перестановка из $n$ элементов, начните с $(n-1)!$):

$$1\times4!+2\times3!+2\times2!+1\times1!+0\times0! = 24 + 12 + 4 + 1 + 0 = 41.$$

Проверьте себя тем, что в конце всегда получается $0\times0!$.

Получилось 41 вместо 42. Причина та же, что и в задаче про перечисление слов. Алгоритм вычисляет номер, начиная считать с нулевого номера. Действительно, для перестановки $12345$ таблица инверсий состоит из нулей, и формула выдает 0. Хотя это первая перестановка. Поэтому к полученному результату нужно не забывать добавлять единицу. Окончательный ответ опять получился 42.

\subsection{Вычисление перестановки по номеру}

Для вычисления перестановки по номеру требуется проделать все действия из предыдущего раздела в обратном порядке. Допустим, нам нужно узнать 426-ую перестановку из перестановок 6 элементов.

Вычитаем единицу, теперь искать надо 425-ую перестановку.

Необходимо разложить 425 в сумму факториалов с коэффициентами. Это можно сделать двумя способами. Первый способ предлагает начать с $5! = 120$ и посмотреть, сколько раз это число входит в 425. Три раза. Итого, $425=3\times5!+\ldots$. Дальше остается $425-360=65$, и нужно определить сколько раз в него входит $4!=24$. Два раза. Итого, $425=3\times5!+2\times4!+\ldots$, в остатке $65-48=17$. Дальше нужно определить, сколько раз в 17 входит число $3!=6$. Дважды.

Продолжая вычисления, можно определить, что
$425=3\times5!+2\times4!+2\times3!+2\times2!+1\times1!+0\times0!$. Соответственно, таблица инверсий получается \verb|322210|.

Другой способ получить представление в виде суммы факториалов~--- заметить, что это фактически представление числа в факториальной системе счисления. Для перевода 425 в факториальную систему нужно последовательно делить его на числа от 1 до 6. Получающиеся остатки~--- это и есть коэффициенты разложения.
$$
\begin{array}{ccccccc}
	425&=&425&\times&1 &+& 0\\
	425&=&212&\times&2 &+& 1\\
	212&=&70&\times&3&+& 2\\
	70&=&17&\times&4&+&2\\
	17&=&3&\times&5&+&2\\
	3&=&0&\times&6 &+& 3
\end{array}.
$$
Таблица инверсий \verb|322210| видна в последнем столбце.

Остается последний шаг, по таблице инверсий получить перестановку.

Итак, имеются шесть чисел $123456$ для перестановки. Таблица инверсий начинается с трех. Отсчитаем три числа и возьмем следующее. Это 4. Значит, наша перестановка начинается на \verb|4|. Меньше четверки как раз будут три числа 1, 2, 3. Больше четверку в перестановке использовать нельзя. Напишем оставшиеся цифры $123\cancel456$.

Следующее число в таблице инверсий~--- 2. Отсчитаем два числа и возьмем следующее. Это 3. Остались следующие числа: $12\cancel3\cancel456$. Теперь наша перестановка начинается на \verb|43|.

Дальше в таблице инверсий опять 2. Отсчитываем два числа, берем следующее, получаем 5. Т.е. справа от пятерки будут отсчитанные только что числа 1 и 2. Перестановка начинается на \verb|435|. Неиспользованные числа: $12\cancel3\cancel4\cancel56$

Продолжая дальше, получаем ответ: 426-ая перестановка это \verb|435621|.

\end{document}