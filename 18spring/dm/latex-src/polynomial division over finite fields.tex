\documentclass{article}

\usepackage[utf8]{inputenc}
\usepackage[russian]{babel}
\usepackage{amsmath}
\usepackage{amsfonts}
\usepackage{framed}
\usepackage{polynomial}

\newenvironment{exercise}{%
\begin{framed}\par\noindent\slshape%
}%
{\end{framed}}

\title{Деление многочленов в полях $\mathbb{Z}_p$}
\author{}
\date{}

\begin{document}
\maketitle

\section{Деление многочленов с вещественными коэффициентами (напоминание)}

Деление многочленов с остатком~--- это операция, аналогичная делению чисел с остатком. Чтобы поделить многочлен $a(x)$ на $b(x)$, нужно найти многочлены $q(x)$ (неполное частное) и $r(x)$ (остаток), чтобы выполнялось соотношение:
$$a(x) = b(x)q(x)+r(x),$$
при этом степень остатка должна быть меньше степени делителя: $\deg r(x) < \deg b(x)$.

Многочлены, как и числа, можно делить в столбик. Традиционную запись деления можно посмотреть, например, в википедии в статье <<деление многочленов столбиком>>. В этом тексте нам будет проще оформлять деление иначе, но в своих работах рекомендуется использовать традиционную запись.

Поделим, например, $x^4$ на $x^2 + 1$:

\begin{alignat*}{7}
x^4\\
x^4&+&x^2&& &=&\mathbf{x^2}&(x^2 + 1)\\
\cline{1-4}
&-&x^2\\
&-&x^2 &-&1 &=&\mathbf{-1}&(x^2 + 1)\\
\cline{3-5}
&& &&1
\end{alignat*}

На первом шаге делитель домножен на $x^2$, результат умножения вычитается из делимого, остается многочлен $-x^2$. Чтобы сократить с ним, делитель домножается на $-1$, и после вычитания остается единица. Это значит, что остаток от деления равен единице, а неполное частное составлено из множителей для делителя, оно равно $x^2-1$. Это можно записать так:
$$x^4=(x^2 + 1)(x^2 - 1) + 1.$$

Другой пример, поделим $6x^4-7x^3+5x^2+3x-1$ на $3x^2-2x+3$:

\begin{alignat*}{11}
6&x^4&-7&x^3&+5&x^2&+3&x&-1\\
6&x^4&-4&x^3&+6&x^2&&&&=&\mathbf{2x^2}&(3x^2-2x+3)\\
\cline{1-6}
&&-3&x^3&-&x^2&+3&x&-1\\
&&-3&x^3&+2&x^2&-3&x&&=&\mathbf{-x}&(3x^2-2x+3)\\
\cline{3-8}
&&&&-3&x^2&+6&x&-1\\
&&&&-3&x^2&+2&x&-3&=&\mathbf{-1}&(3x^2-2x+3)\\
\cline{5-10}
&&&&&&4&x&+2\\
\end{alignat*}


Итого, $6x^4-7x^3+5x^2+3x-1=(3x^2 - 2x+3)(2x^2-x-1)+(4x+2)$.

\section{Деление в поле $\mathbb{Z}_p$}

В предыдущем разделе многочлены имели в качестве коэффициентов вещественные числа. Теперь множество коэффициентов многочлена~--- это поле $\mathbb{Z}_p$. Фактически, это целые числа по модулю $p$. Другими словами, в качестве коэффициентов многочлена можно использовать только целые числа, причем числа, сравнимые по модулю $p$, считаются одинаковыми. Например, по модулю 7, многочлены $x^2 + 2x + 3$ и $-6x^2+16x+24$~--- это одинаковые многочлены.

Поделим $3x^4+5x^3+x+3$ на $5x^2+2x+1$ в $\mathbb{Z}_7$, т.е. по модулю 7. Первым шагом нужно подобрать множитель для делителя, чтобы совпали первые одночлены. Другими словами, нужно умножить что-то на $5x^2$, чтобы получить $3x^4$. В случае вещественных чисел домножать нужно на $\frac35x^2$, но числа $\frac35$ просто нет в поле $\mathbb{Z}_7$. Вспомним, что мы выполняем вычисления по модулю 7, и тогда домножить можно на 2, действительно, $2\times 5x^2=10x^2$, а это то же самое, что $3x^2$ по модулю 7.

Теперь можем написать весь процесс деления.

\begin{alignat*}{11}
3&x^4&+5&x^3&&&+&x&+3\\
3&x^4&+4&x^3&+2&x^2&&&&=&\mathbf{2x^2}&(5x^2+2x+1)\\
\cline{1-6}
&&&x^3&+5&x^2&+&x&+3\\
&&&x^3&+6&x^2&+3&x&&=&\mathbf{3x}&(5x^2+2x+1)\\
\cline{3-8}
&&&&6&x^2&+5&x&+3\\
&&&&6&x^2&+&x&+4&=&\mathbf{4}&(5x^2+2x+1)\\
\cline{5-10}
&&&&&&4&x&+6\\
\end{alignat*}


Обратите внимание, что в вычислениях используются коэффициенты многочлена только из диапазона от 0 до 6 (в общем случае от 0 до $p-1$). Все числа приводятся в этот диапазон. Например, при первом же вычитании из $0x^2$ вычитается $2x^2$, и здесь можно было бы написать ответ $-2x^2$, но он сразу превращен в $5x^2$ по модулю 7. Причина в том, что поле $\mathbb{Z}_p$, как принято считать, состоит только из возможных остатков по модулю $p$, поэтому числа $-2$ в поле не существеует. Мы только понимаем, что $-2$ это другая форма записи числа $5$.

Ответ получился следующий: остаток $4x+6$, неполное частное~--- $2x^2+3x+4$. Его можно проверить следующим равенством:
$$3x^4+5x^3+x+3=(5x^2+2x+1)(2x^2+3x+4) + (4x+6).$$
Равенство верное, потому что после раскрытия скобок в правой части получается $10x^4+19x^3+28x^2+15x+3$, а это то же самое по модулю 7, что и левая часть.

\end{document}