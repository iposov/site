\documentclass{article}

\usepackage[utf8]{inputenc}
\usepackage[russian]{babel}
\usepackage{amsmath}
\usepackage{hyperref}
\usepackage{framed}
\renewcommand{\c}[2]{$C_{#1}^{#2}$}
\newcommand{\cc}[2]{C_{#1}^{#2}}

\newenvironment{exercise}{%
\begin{framed}\par\noindent\slshape%
}%
{\end{framed}}

\title{Комбинаторная вероятность}
\author{}
\date{}

\begin{document}
\maketitle

\section{Определение}
Вы еще не изучали курс «Теории вероятностей», поэтому забудьте всё, что вы знаете о теории вероятности. Для решения задач нам будет достаточно только одного определения комбинаторной вероятности.

Чтобы было проще понять определение, начнем с простейших задач.

\begin{exercise}
1. Если бросить кубик, с какой вероятностью на нем выпадет единица?
\end{exercise}

\begin{exercise}
2.	Если бросить кубик, с какой вероятностью на нем выпадет простое число?
\end{exercise}

\begin{exercise}
3.	Если бросить два кубика, с какой вероятностью на них в сумме выпадет 7?
\end{exercise}

Для решения нужно понять, какой мы совершаем эксперимент, и определиться с множеством возможных исходных эксперимента. В первых двух задачах мы бросаем один кубик, и в результате это эксперимента может получиться один из шести исходов: $\{1, 2, 3, 4, 5, 6\}$.

В третьей задаче мы бросаем два кубика. Каждый исход~--- это пара значений, выпавших на кубиках. Первое значение это число, выпавшее на первом кубике, и оно имеет шесть вариантов. Второе значение, аналогично, это число, выпавшее на втором кубике, и у него тоже есть шесть вариантов. Отсюда $6\times6=36$ возможных исходов эксперимента.

Дальше, из всех возможных исходов нужно выделить те, про которые спрашивается в задаче. В первой задаче интересен только один исход $\{1\}$. Во второй задаче~--- три: $\{2, 3, 5\}$. В последней задаче~--- семь: $$\{(1,6),\ (2, 5),\ (3, 4),\ (4, 3),\ (5, 2),\ (6, 1)\}.$$

\subsection{Определение комбинаторной вероятности}

Вероятность события в эксперименте~--- это дробь, где в знаменателе находится количество всех возможных исходов эксперимента, а в числителе~--- количество исходов, при которых событие наступило.

Таким образом, ответ в первой задаче:

$$\dfrac16.$$

Во второй:

$$\dfrac36.$$

В третьей:

$$\dfrac7{36}.$$

Как и в других задачах по комбинаторике, рекомендуется не упрощать, не вычислять и не сокращать ответ. Т.е. лучше оставьте ответ $\frac36$, и не приводите его к $\frac12$, чтобы по ответу можно было понять, как вы решали.

\section{Носки в ящике}

\begin{exercise}
	В ящике лежат 10 чистых синих носок и 20 чистых красных носков. С какой вероятностью два случайно вытащенных носка будут разного цвета?
\end{exercise}

Для решения вначале нужно определиться с тем, какой мы проводим эксперимент. В этой задаче эксперимент можно проводить по-разному, и нужно явно выбрать один из нескольких вариантов. Мы можем либо достать два носка одновременно, либо мы можем достать сначала один, а потом второй. Решение в обоих случаях будет разным, но, к счастью, ответы в обоих случаях совпадут.

\subsection{Достаём два носка одновременно}

Сколько есть способов вытащить два носка из $10 + 20 = 30$? Это, фактически, вопрос, сколькими способами можно выбрать 2 объекта из 30. Т.е. ответ \c{30}2. Это будет знаменатель в ответе.
А числитель? Нужно разобраться, в каких случаях мы получаем два носка разных цветов. Нас не интересует порядок, в котором мы доставали носки, мы только знаем, что у нас должен быть один синий (10 вариантов) и один красный (20 вариантов). Т.е. мы всего имеем $10\times20$ комбинаций того, какие носки можно было достать. Это приводит к ответу:
$$\dfrac{10\cdot20}{\cc{30}2}.$$

\subsection{Достаём два носка по очереди}

Знаменатель дроби~--- это $30\times29$, потому что мы сначала достаем один носок из 30, потом один из 29 оставшихся.

Для числителя нужно посчитать, в скольких случаях мы получаем два носка разных цветов. Мы достаем носки по очереди, поэтому нам подходят ситуации, когда мы сначала достаем синий носок, а потом красный. И когда мы сначала достаем красный, а потом синий. Вариантов синего-потом-красного будет $10\times20$, а вариантов красного-потом-синего будет $20\times10$. Итого, ответ в задаче:

$$\dfrac{10\cdot20 + 20\cdot10}{30\cdot29}.$$

Убедитесь, что численно ответы в обоих случаях совпадают. Самая распространенная ошибка при решении подобной задачи, это взять числитель из одного вида эксперимента, а знаменатель из другого. Получается что-то типа $\dfrac{10\cdot20}{30\cdot29}$, что неправильно. Поэтому однозначно определяйтесь вначале, достаете вы одновременно или по очереди.

\section{Переход к дополнению}

\begin{exercise}
	В ящике лежат 10 чистых синих носок и 20 чистых красных носков. С какой вероятностью два случайно вытащенных носка будут \textbf{одного} цвета?
\end{exercise}

В условии задачи эксперимент не изменился, но изменились те исходы, которые мы считаем успешными, т.е. те, которые нужно посчитать для числителя. Эту задачу можно свести к предыдущей. Все возможные исходы эксперимента делятся на те, в которых мы достали носки одинаковых цветов, и те, в которых мы достали носки разных цветов.

Для определенности считаем, что достаем носки одновременно. Всего исходов \c{30}2, и из них $10\cdot20$ это исходы с носками разного цвета. Поэтому, исходов с носками одного цвета будет $\cc{30}{2}-10\cdot20$, и окончательный ответ:

$$\dfrac{\cc{30}{2}-10\cdot20}{\cc{30}{2}}.$$

\begin{exercise}
	Этот ответ соответствует экспериментам с одновременным доставанием носок. А какой будет ответ, если доставать носки по очереди?
\end{exercise}

Обычно, к дополнению нужно переходить, если не удается решить задачу напрямую. Но эту задачу можно было бы решить и без перехода к дополнению. Тогда ответами было бы:

$$\dfrac{\cc{20}{2} + \cc{10}{2}}{\cc{30}{2}},$$

если считать, что носки достаются одновременно (надо либо два синих, либо два красных), или

$$\dfrac{20\cdot19 + 10\cdot9}{30\cdot29},$$

если считать, что носки достаются по очереди. Здесь опять в числителе разбираются случаи того, что два носка синих или два носка красных.
Проверьте, что все три ответа в задаче численно совпадают.
\end{document}