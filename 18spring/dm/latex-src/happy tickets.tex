\documentclass{article}

\usepackage[utf8]{inputenc}
\usepackage[russian]{babel}
\usepackage{amsmath}
\usepackage{hyperref}

\newenvironment{exercise}{%
\begin{framed}\par\noindent\slshape%
}%
{\end{framed}}

\title{Задача о счастливых билетах}
\author{}
\date{}

\begin{document}
\maketitle

\section{Постановка задачи}
На лекции вы наверняка разбирали задачу о счастливых билетах. Здесь не будет повторения лекции, а будет разобрана аналогичная задача с другим определением того, что такое счастье для билета.

Предположим, что билеты состоят из пяти цифр, и счастливыми считаются те, у которых сумма первых трех цифр на 5 больше суммы последних двух цифр. Для номеров билетов нет запрета на то, чтобы ноль был первой цифрой. Например, счастливыми будут билеты:

\begin{tabular}{lcc}
	номер&суммы цифр&+ 5\\
	\hline
	\verb|12512|:&1 + 2 + 5 = 1 + 2& + 5 \\
	\verb|23927|:&2 + 3 + 9 = 2 + 7& + 5 \\
	\verb|12200|:&1 + 2 + 2 = 0 + 0& + 5
\end{tabular}

Вопрос. Сколько всего существует счастливых билетов? Эту задача и будет далее решена.

Всего билетов есть $10^5$, значит, ответ получится не больше этого числа.

Эту задачу удобно превратить в другую задачу с более естественным условием. Вместо поиска счастливых билетов удобней искать билеты с фиксированной суммой цифр. Следующий раздел демонстрирует, как одна задача преобразуется в другую.

Решение новой задачи о билетах с фиксированной суммой цифр можно проводить двумя способами. Вам потребуется воспользоваться обоими способами, а потом убедиться, что они привели вас к одинаковым ответам.

\section{Преобразование задачи в задачу о сумме цифр}
Сначала научимся дополнять цифры до девяти:

\begin{center}
\begin{tabular}{ccc}
	цифра&$\leftrightarrow$&дополнение\\
	\hline
	0&$\leftrightarrow$&9\\
	1&$\leftrightarrow$&8\\
	2&$\leftrightarrow$&7\\
	3&$\leftrightarrow$&6\\
	4&$\leftrightarrow$&5\\
	5&$\leftrightarrow$&4\\
	6&$\leftrightarrow$&3\\
	7&$\leftrightarrow$&2\\
	8&$\leftrightarrow$&1\\
	9&$\leftrightarrow$&0\\
	$d$&$\leftrightarrow$&$9-d$
\end{tabular}
\end{center}

Сопоставим каждому билету, неважно, счастливому или нет, другой билет по следующему правилу: первые три цифры не меняются, четвертая и пятая цифры дополняются до 9. Например, билет \verb|12512| превращается в билет \verb|12587|.

В общем виде, билет $\overline{abcde}$ превращается в
$\overline{abc(9-d)(9-e)}$. Это сопоставление является взаимно однозначным сопоставлением множества билетов в себя, это значит, что каждому билету однозначно сопоставлен какой-то другой билет.

Рассмотрим сумму цифр сопоставленного билета: $a + b + c + 9 - d + 9 - e = 18 + a + b + c - (d + e)$. Для счастливых, и только для счастливых билетов последнее выражение равно $18 + 5 = 23$.

Другими словами, каждому счастливому билету сопоставлен билет с суммой цифр 23.

Так как сопоставление взаимно однозначно, можно сделать вывод, что количество счастливых билетов равно количеству билетов с суммой цифр 23. В следующих разделах мы будем считать количество билетов с этой суммой и забудем про счастливые билеты совсем.

Заметим, что сопоставление можно было провести по-другому:
билету $\overline{abcde}$ можно сопоставить $\overline{(9-a)(9-b)(9-c)de}$. Т.е. дополнить до 9 первые цифры, а не последние. В это случае сумма цифр полученного билета была бы равна
$9-a+9-b+9-c+d+e=27-(a+b+c)+(d+e)$. И для счастливых билетов это равно $27-5=22$. Итого, количество счастливых билетов равно количеству билетов с суммой цифр 22. Перед этим мы получали другое число 23, но это не проблема. Получается, что билетов с суммой цифр 22 столько же, сколько билетов с суммой 23.

\section{Символьные вычисления}
Рассмотрим многочлен

$$(x^0+x^1+x^2+x^3+\cdots+x^9)^5=(1+x+x^2+x^3+\cdots+x^9)^5.$$

Здесь в скобках в качестве степеней написаны все возможные цифры номера от нуля до 9. Цифры именно такие, потому что в задаче подразумевается десятичная система счисления. Иначе в задаче явно было бы сказано про другую систему счисления. Указанное в степени число 5~--- это количество цифр в номере.

Если немного подумать, оказывается, что после раскрытия скобок в этом многочлене, коэффициент при $x^{22}$ и будет ответом в задаче. Вручную раскрыть скобки слишком сложно, лучше воспользоваться системой численных вычислений. Например, такие выичсления делает сайт \url{wolframalpha.com}.

Введите в запросе команду\\
\verb|expand (x^0+x^1+x^2+x^3+x^4+x^5+x^6+x^7+x^8+x^9)^5|\\
В ней слово expand означает просьбу раскрыть скобки.
В качестве ответа будет приведен многочлен, в котором можно увидеть коэффициенты при степенях 22 и 23: $\cdots+6000x^22+6000x^23$. Как и предполагалось, ответы для 22 и 23 должны были совпасть.

Итого, 6000~--- ответ в исходной задаче. Многочлен показал, что есть 6000 номеров с суммой цифр 22 (или 23), а мы знаем, что таких номеров столько же, сколько счастливых номеров.

\section{Вычисление по формуле включений-исключений}

Задача вычисления количества номеров с заданной суммой цифр очень похожа на известную нам задачу: вычисление количества разложений числа на слагаемые:
$$x_1+x_2+x_3+x_4+x_5=22, x_i\ge 0.$$
Ответ в этой задаче будет $C_{22+4}^4$. Численно это равно $14950$, что сильно больше правильного ответа $6000$. Так получается, потому что не учитывается еще одно условие что все $x_i<10$.

Из ответа $C_{22+4}^4$ необходимо вычесть все неправильные номера, имеющие цифру хотя бы 10.

Разобъем неправильные номера на пять множеств: $A_1$~--- это те номера, у которых первая цифра хотя бы 10. $A_2$~--- это те номера, у которых вторая цифра хотя бы 10 и т.д. Все неправильных номера~--- это объединение множеств $A_1\cup A_2\cup A_3\cup A_4\cup A_5$. Для окончания решения необходимо найти количество неправильных номеров, т.е. мощность указанного множества. Для этого подходит формула включений-исключений:

$$
\begin{gathered}
|A_1\cup A_2\cup A_3\cup A_4\cup A_5|=\\
|A_1|+|A_2|+|A_3|+|A_4|+|A_5|-\\
|A_1\cap A_2|-|A_1\cap A_3|-\ldots-|A_4\cap A_5|+\\
|A_1\cap A_2 \cap A_3|-|A_1\cap A_2\cap A_4|-\ldots-|A_3\cap A_4\cap A_5|-\\
\ldots
\end{gathered}
$$

Разберем формулу постепенно. В первой строке написана мощность множества неправильных номеров, это ровно то, что нужно вычислить.

Во второй строке написана сумма мощностей множеств $A_i$, это количество номеров, у которых $i$-ая цифра хотя бы 10. Посчитаем, например, $|A_1|$, это количество номеров, у которых первая цифра хотя бы 10, т.е. необходимо решить задачу:

$$x_1+x_2+x_3+x_4+x_5=22, x_1\ge10, x_i\ge 0.$$

Вычитая 10 из обеих частей, получаем другую задачу:

$$y_1+x_2+x_3+x_4+x_5=12, y_1\ge0, x_i\ge 0.$$

Здесь ответ $C_{12+4}^4$. Во второй строке формулы включений-исключений складывается пять одинаковых слагаемых, поэтому там написано $5C_{12+4}^4$.

Переходим к третьей строчке формулы. В ней складываются все попарные пересечения множеств. Пересечение, например, $A_1$ и $A_2$ это все номера, у которых неправильная и первая, и вторая цифры. Т.е. и первая, и вторая цифры хотя бы 10. Чтобы найти количество таких номеров, надо решить задачу:

$$x_1+x_2+x_3+x_4+x_5=22, x_1\ge10, x_2\ge10, x_i\ge 0.$$

Вычитаем 20 из обеих частей, получаем задачу 

$$y_1+y_2+x_3+x_4+x_5=2, y_1\ge0, y_2\ge0, x_i\ge 0.$$

В ней ответ $C_{2 + 4}^4$. Всего слагаемых в третьей строчке формулы~--- $C_5^2$, потому что необходимо выбрать все пары множеств из пяти множеств. Соответственно, в третьей строчке формулы написано $C_5^2C_{2 + 4}^4$.

В четвертой строчке формулы перебираются пересечения множеств по три. В пересечении трех множеств находятся номера, у которых есть три цифры, каждая из которых хотя бы 10. Но в таком случае сумма цифр будет хотя бы 30, и она не может быть равна 22. Поэтому в строках формулы включений-исключений, начиная с четвертой строки написаны нули.

Итого, формула включений-исключений дает результат, что неправильных номеров $5C_{12+4}^4-C_5^2C_{2 + 4}^4$. Для получения окончательного ответа в задаче нужно из количества всех номеров вычесть количество неправильных: $C_{22+4}^4-5C_{12+4}^4+C_5^2C_{2 + 4}^4$.

Полученное выражение можно вычислить численно, и результат будет как раз 6000. Это гарантирует, что ответ в задаче вычислен правильно, т.к. два совершенно разных метода дали один и тот же результат.

Последние замечания. Было показано, что задачу можно решать для суммы цифр 23 и для суммы цифр 22. В обоих случаях ответ получился бы одинаковый, 6000. В своей задаче вы можете тоже определить, для каких двух сумм ее надо решать, и решать после этого для меньшей суммы. Для меньшей суммы чаще получается проще.

При сдаче задачи обязательно решите ее двумя предложенными способами. Тем более, что первый очень простой, в нем все вычисления за вас делаются автоматически.

\end{document}